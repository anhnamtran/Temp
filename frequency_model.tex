\subsection{The Problem}
In both the original model and the maintenance cost model, there was no way for
the user to specify the importance (or frequency of usage) of a query or a
mutator, this could be useful for specifications that could be implemented in
several different ways, each way more efficient for different queries or
mutators. For example, this specification:

\begin{lstlisting}
    Product:
        state xs : Bag<Int>
        state ys : Bag<Int>

        query product()
            [(x, y) | x <- xs, y <- ys]

        op add_x(x : Int)
            xs.add(x);
\end{lstlisting}

Can produce a state that is a list of all \code{(x, y)} tuples. This solution is
inefficient for the mutator but efficient for the query. When an \code{x} is
added, the list of tuples will have to grow in size by the cardinality of the bag
\code{ys}, which is problematic if \code{ys} gets really big. In the original
and the maintenance cost model, there is a heuristic in place to prevent Cozy
from outputting this solution. However, this solution is more efficient if
the query is used significantly more than the mutator. If we allow the user to
specify this information, not only will Cozy have one less heuristic, it will
also produce implementations that are more tailored to how they will be used.

\subsection{Overview}
The frequency cost model uses part of the maintenance cost model but replaces
the unprioritized ordering with a prioritized ordering of frequency cost,
asymptotic runtime, max storage, true runtime, and size.

The frequency cost of an expression is the weighted sum of its true runtime and
its maintenance cost for every mutator that is available. The true runtime is
considered to be the runtime of the query. The weights are the frequency of the
query and mutators.

\subsection{Challenges}
There are a few noticeable challenges with this model. The first is since this
model is optimizing for multiple objectives, there might be an issue with just
taking the weighted sum of runtime and maintenance costs. This could possibly
produce confusing results in the future, especially for more complicated
specifications.

Moreover, since the model is using parts of the maintenance cost model, it is
also slower than the original cost model. However, since it makes fewer solver
calls, it is slightly faster than the maintenance cost model.

The final issue that this model faces is the scale of the frequencies. Currently,
there are no specific guidelines on how to use the frequencies, or what numbers
the frequencies should be. Manual testing on the example above shows that if the
query has a frequency that is at least $\times 25$ the frequency of the mutator,
Cozy will produce a different implementation. However, it is unclear if the
scale is the same for other specifications, especially those that more
complicated.
